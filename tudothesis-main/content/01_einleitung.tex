\chapter{Einleitung}

Die Anwendung von Künstlichen Intelligenzen (KI) und Maschinellem Lernen (ML) findet in der heutigen Zeit immer mehr Anwendung.
So auch im Medizinischen Bereich beispielsweise zur Erkennung von Krankheiten.
In dieser Arbeit liegt der Fokus auf das erkennen von Hirntumoren mithilfe eine Klassifizierungsalgorithmus.

Für die Diagnostik eines Hirntumors, wird zuerst eine Magnetresonanztomographie (MRT) durchgeführt.
Zur spezifischen Bestimmung wird in der Regel eine Biopsie veranlasst.
Da es sich hierbei um einen Operativen Eingriff handelt, birgt dieser, wie alle Operative Eingriffe auch, gewisses Risiken
Klassifizierungsalgorithmen könnten dabei helfen, die Notwendigkeit einer Biopsie zu reduzieren und somit das Risiko für die Patient:innen zu verringern.
Für die Entwicklung solcher Algorithmen, werden ausreichend Daten benötigt, damit sie ein zuverlässige Aussage über die Art des Tumors treffen können. 
Aus Datenschutzrechtlichen Gründen, sind diese Datenmengen jedoch nicht ausreichend vorhanden. 
Zudem ist es bei seltenen Tumorarten noch mal schwieriger genügend Daten zu sammeln.
Ziel dieser Arbeit ist es daher zu untersuchen, welchen Einfluss die Größe des Datensatzes auf die Leistungsfähigkeit eines Klassifikationsalgorithmus hat. 
Zudem wird ermittelt, wie groß ein Datensatz mindestens sein muss um zuverlässige Vorhersagen zu treffen.

Zu Beginn wird in der Theorie einen Überblick über die Entstehung der verwendeten MRT-Bilder gegeben. 
Anschließend werden die Grundlagen des Maschinellen Lernen behandelt.
Diese beinhalten den Aufbau und das Training von Neuronalen Netzwerken, Convolutional Neural Network, das in der Arbeit verwendet wird.
Im Anschluss wird der Aufbau des in dieser Arbeit verwendete Convolutional Neural Netzwerk erklärt und der Aufbau des Datensatzes. 
Es wird in der Arbeit zwischen zwei Klassifikationen unterschieden. 
Zum einen werde die folgenden Untersuchungen für die Klassifikation zwischen no Tumor und Tumor, sowie zwischen Glioma und Meningioma durchgeführt.

Der Einfluss der Datensatzgröße wird untersucht, indem die Anzahl an Trainingsdaten schrittweise reduziert wird.
Zur Beurteilung der Leistung wird die Accuracy, Sensitivity und Specificity berechnet wird. 
Darüber hinaus wird überprüft, ob Datenaugmentation die Leistung verbessern kann.
Zusätzlich wird die Leistung betrachtet, wenn nur eine der beiden Klasse reduziert wird. 
Zum Abschluss werden die Ergebnisse der verschiedenen Untersuchungen diskutiert.