\chapter{Einleitung}

Die Anwendung von Künstlichen Intelligenzen (KI) und Maschinellem Lernen (ML) findet in der heutigen Zeit immer mehr Anwendung.
So auch im Medizinischen Bereich, beispielsweise zur Erkennung von Krankheiten.
In dieser Arbeit liegt der Fokus auf das Erkennen von Hirntumoren mithilfe eines Klassifizierungsalgorithmus.

Für die Diagnostik eines Hirntumors, wird zuerst eine Magnetresonanztomographie (MRT) durchgeführt.
Zur spezifischen Bestimmung wird in der Regel eine Biopsie veranlasst.
Da es sich hierbei um einen Operativen Eingriff handelt, birgt dieser, wie alle operative Eingriffe auch, gewisse Risiken.
Klassifizierungsalgorithmen könnten dabei helfen, die Notwendigkeit einer Biopsie zu reduzieren und somit das Risiko für die Patient:innen zu verringern.
Für die Entwicklung solcher Algorithmen, werden ausreichend Daten benötigt, damit sie eine zuverlässige Aussage über die Art des Tumors treffen können. 
Aus datenschutzrechtlichen Gründen sind diese Datenmengen jedoch nicht ausreichend vorhanden. 
Zudem ist es bei seltenen Tumorarten noch mal schwieriger, genügend Daten zu sammeln.
Ziel dieser Arbeit ist es daher zu untersuchen, welchen Einfluss die Größe des Datensatzes auf die Leistungsfähigkeit eines Klassifikationsalgorithmus hat. 
Zudem wird ermittelt, wie groß ein Datensatz mindestens sein muss um zuverlässige Vorhersagen zu treffen.

Zu Beginn wird in der Theorie ein Überblick über die Entstehung der verwendeten MRT-Bilder gegeben. 
Anschließend werden die Grundlagen des Maschinellen Lernen behandelt.
Diese beinhalten den Aufbau und das Training von Neuronalen Netzwerken, insbesondere von Convolutional Neural Networks, die in dieser Arbeit verwendet werden.
Im Anschluss wird der Aufbau des in dieser Arbeit verwendeten Convolutional Neural Networks erklärt und der Aufbau des Datensatzes. 
Es wird in der Arbeit zwischen zwei Klassifikationen unterschieden. 
Zum einen werden die folgenden Untersuchungen für die Klassifikation zwischen no Tumor und Tumor, sowie zwischen Glioma und Meningioma durchgeführt.

Der Einfluss der Datensatzgröße wird untersucht, indem die Anzahl an Trainingsdaten schrittweise reduziert wird.
Zur Beurteilung der Leistung werden die Accuracy, Sensitivity und Specificity berechnet wird. 
Darüber hinaus wird überprüft, ob eine Datenaugmentation die Leistung verbessern kann.
Zusätzlich wird die Leistung betrachtet, wenn nur eine der beiden Klassen reduziert wird. 
Zum Abschluss werden die Ergebnisse der verschiedenen Untersuchungen diskutiert.