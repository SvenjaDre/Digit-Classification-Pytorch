\chapter{Einleitung}

Für die Diagnostik eines Hirntumors, wird als erster Schritt ein MRT veranlasst und im zweiten Schritt eine Biopsy 
durchgeführt, um die spezifische Tumorart zu bestimmen.
Da es sich jedoch bei der Biopsy um einen Operativen Eingriff handelt, birgt diese auch ein gewisses Risiko.
Mithilfe von Klassifizierungsalgorithmen soll es möglich sein, dass der Schritt der Biopsy nicht notwendig ist 
und somit das Risiko eines Eingriffs  zu verringern.
Für die Verwendung solcher Algorithmen, werden ausreichend Daten benötigt, um diese genügend zu trainieren, damit die 
Voraussage zuverlässig sind. 
Aus Datenschutzrechtlichen Gründen, sind diese Datenmengen nicht ausreichend vorhanden. Zudem ist es bei seltenen
Tumoren noch mal schwieriger genügend Daten zu sammeln.
Diese Arbeit beschäftigt sich damit, wie viele Daten es braucht, damit ein Klassifizierungsalgorithmus gut trainiert werden
kann und Zuverlässige aussagen Betrifft. 
Dafür wird ein Convolutional Neural Network trainiert und getestet.

