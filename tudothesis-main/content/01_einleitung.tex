\chapter{Einleitung}

Die Anwendung von Künstlichen Intelligenzen und Maschinellem Lernen findet in der heutigen Zeit immer mehr Anwendung.
So auch in der Medizin für unter anderen die Erkennung von Krankheiten.
In dieser Arbeit wird sich auf das erkennen von Hirntumoren mittels eine Klassifizierungsalgorithmus Konzentriert.
Für die Diagnostik eines Hirntumors, wird als erster Schritt ein MRT veranlasst und im zweiten Schritt eine Biopsie 
durchgeführt, um die spezifische Tumorart zu bestimmen.
Da es sich jedoch bei der Biopsie um einen Operativen Eingriff handelt, birgt diese auch ein gewisses Risiko.
Mithilfe von Klassifizierungsalgorithmen soll es möglich sein, dass der Schritt der Biopsie nicht notwendig ist 
und somit das Risiko eines Eingriffs zu verringern.
Für die Verwendung solcher Algorithmen, werden ausreichend Daten benötigt, um diese genügend zu trainieren, damit die 
Voraussage zuverlässig sind. 
Aus Datenschutzrechtlichen Gründen, sind diese Datenmengen jedoch nicht ausreichend vorhanden. Zudem ist es bei seltenen
Tumoren noch mal schwieriger genügend Daten zu sammeln.
Diese Arbeit beschäftigt sich damit, wie viele Daten es braucht, damit ein Klassifizierungsalgorithmus gut trainiert werden
kann und Zuverlässige aussagen Betrifft. 
Zu Beginn wird in der Theorie einen Überblick über die Entstehung der MRT-Bilder und die Grundlagen des Maschinellen Lernen gegeben.
Zudem wird der Aufbau und das trainieren eines Neuronalen Netzwerkes erläutert.
Nach dem theoretischen Teil, wird der Aufbau des, in dieser Arbeit verwendete, Convolutional Neural Netzwerk erklärt und der Aufbau des Datensatzes.
Es wird das Netzwerk für 
Für die Bewertung der Leistungsfähigkeit wird ein Convolutional Neural Network trainiert und anschließend getestet.
Dabei wird zur Bewertung die Accuracy, Sensitivity und Specificity betrachtet.

