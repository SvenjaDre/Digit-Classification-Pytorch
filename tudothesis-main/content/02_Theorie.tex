\chapter{Theorie}

\section{Grundlagen MRT}
Die Magnetresonanztomographie (MRT) ist ein Bildgebendes Verfahren, bei der keine ionisierender Strahlung eingesetzt wird.
Dadurch gibt es keine Strahlenbelastung für Patientinnen und Patienten und die Untersuchung kann beliebig oft wiederholt werden.
Die MRT wird zur Darstellung von morphologischen Strukturen und zur Abbildung funktioneller Prozesse verwendet.
Das MRT-Gerät besteht aus einem zylinderförmigen, supraleitenden Magneten, der ein homogenes Hauptmagnetfeld $B_0$ erzeugt. %Shimming optional
In der klinischen Anwendung liegt die Magnetfeldstärke bei $\qty{1.5}{T}$ - $\qty{3}{T}$.~\cite{Schlegel}
Die Kernteilchen besitzen einen magnetischen Moment $\vec{m}$, aufgrund dessen das sie um ihre eigene Achse, mit der Lamorfrequenz 
\begin{equation}
    \omega_0 = \gamma \cdot B_0,
\end{equation}
präzedieren. $\gamma$ beschriebt dabei das Gyromagnetische Verhältnis, welches Materialabhängig ist.
Die magnetischen Momente summieren sich im Kern und es entsteht eine gesamt Magnetisierung $\vec{M}$ des Atoms. Befinden sich diese Atome im Magnetfeld,
richten sich die Spins der Kernteilchen parallel und antiparallel nach dem Magnetfeld aus. Sie befinden sich im thermischen Gleichgewicht
Bei Atomen mit einer geraden Anzahl an Kernteilchen hebt sich beim aufsummieren die magnetischen Momente auf und besitzen somit keine
Magnetisierung. Diese ist jedoch notwendig für die Messung eines Signals im MRT.
Somit können nur Atome mit einer ungeraden Anzahl an Kernteilchen gemessen werden. 
Am häufigsten wird zur Messung Wasserstoff verwendet, da dies mit 80$\%$ am häufigsten im Menschlichen Körper vorkommt.
Mittels eines Hochfrequenzsignals (HF) werden die Spins um einen Winkel
\begin{equation}
    \alpha = \gamma \cdot B_1 \cdot T 
\end{equation}
gekippt. Dabei entspricht $B_1$ dem Anregungsimpuls und $T$ die Zeit des HF-Impulses.
Die Frequenz des HF-Signals entspricht der Lamorfrequenz des Atoms. 
Durch die Änderung der Magnetisierung, mittels des HF-Signals, lässt sich ein Signal messen. 
Beim Ausschalten des Signals, relaxieren die Kernteilchen zurück in ihre Gleichgewichtsmagnetisierung. 
Dabei lässt sich zwischen der T1-Relaxation und der T2-Relaxation. 
Bei der T1- Relaxation wird die Längsmagnetisierung betrachte, welche mit der Zeit zunimmt.
% Bild 
Die T2-Relaxation beschreibt die Quermagnetisierung, welche mit der Zeit abnimmt, da die Spins dephasieren.~\cite{Pollmann}


\section{Maschinelles Lernen}
Supervised lernen (unsupervised)


\subsection{Neuronales Netzwerk}

Aufbau
    weights 
training: loss fkt reduzieren
Hyperparameter
Aktivierungsfunktion 
\subsection{Convolutional Neural Network}

Aufbau
