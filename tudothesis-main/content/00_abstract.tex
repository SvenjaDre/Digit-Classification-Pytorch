\thispagestyle{plain}

\section*{Kurzfassung}
In dieser Arbeit wird die Leistungsfähigkeit von Klassifizierungsalgorithmen bei der Erkennung von Hirntumoren bei unterschiedlicher Datensatzgrößen untersucht.
Zunächst werden geeignete Hyperparameter für ein Convolutional Neural Networks ermittelt, indem verschiedene Wertebereiche getestet werden.
Anschließend wird die Anzahl der Trainingsdaten schrittweise reduziert, wobei die Modellleistung anhand von Accuracy, Sensitivity und Specificity bewertet wird.
Die Untersuchung erfolgt für zwei Klassifikationen. 
Einmal die Unterscheidung zwischen Tumor und no Tumor und einmal zwischen Glioma und Meningioma.
Die Ergebnisse zeigen, dass für die Klassifikation von Tumor und no Tumor mindestens 2042 Training samples für eine stabile Leistung.
Für die Unterscheidung Glioma und Meningioma konnte keine eindeutige Sättigung festgestellt werden, jedoch deutet sich eine stabile Leistung bei 1915 Training samples an.
Der Einfluss von Datenaugmentation wird untersucht und zeigt eine geringe Leistungssteigerung für sie Tumor und no Tumor Unterscheidung.
Für die Differenzierung zwischen den zwei Tumorarten wurde keine signifikante Leistungsverbesserung festgestellt.
Zudem wurde der Einfluss auf die Leistung untersucht bei der Reduzierung einer Klasse.
Die Reduktion einer Klasse zeigt eine verschlechterte Leistung, wodurch sich sagen lässt das eine relative ausgeglichener Datensatz für eine stabile Leistung ist. 

\section*{Abstract}
\begin{foreignlanguage}{english}
This thesis investigates the performance of classification algorithms for brain tumor detection with varying dataset sizes.
First, suitable hyperparameters for a Convolutional neural network (CNN) are determined by testing different parameter ranges.
Then, the number of training samples is gradually reduced and the performance is evaluated by calculating the accuracy, sensitivity, and specificity.
The study covers two classification tasks. The first is distinguishing between tumor and no tumor, and the second between glioma and meningioma.
The results show that at least 2042 training samples are needed for stable performance in the tumor and no tumor classification.
For the differentiation between glioma and meningioma,was no clear saturation observed, but stable performance appears to start around 1915 training samples.
The effect of data augmentation was also examined and its showing a slight improvement in performance for the tumor ans no tumor classification.
No significant performance gain was found for the differentiation between the two tumor types.
Furthermore, the impact of class imbalance due to reduction of one class was examined.
Reducing one class led to decreased model performance, indicating that a relatively balanced dataset is important for reliable performance.
\end{foreignlanguage}
