\chapter{Methodik}

\section{Datensatz}
Für diese Arbeit wurde Datensatz "Brain Tumor MRI Dataset", welcher auf Kaggle veröffentlicht wurde, verwendet.~\cite{msoud_nickparvar_2021}
Dieser beinhaltet die vier Klassen: no Tumor, Glioma, Meningioma und Pituitary.
In dieser Arbeit wird die Klasse Pituitary nicht betrachtet.
Aufgeteilt ist der Datensatz in Trainigsdaten und Testdaten.
Die Anzahl der verwendet samples sind in der Tabelle \ref{tab:daten} dargestellt. 
\begin{table}[H]
    \centering
    \begin{tabular}{l c r}
        \hline
        Klasse      & Trainingssamples & Testsamples \\
        \hline
        no Tumor    &    1595              & 405 \\
        Glioma      &    1321              & 300 \\
        Meningioma  &    1339              & 306 \\
        \hline
  \end{tabular}
  \caption{Anzahl der verwendeten Trainigssamples und Testsamples.}
  \label{tab:daten}
\end{table}
Im Folgenden wurde zwei unterschiedliche Klassifikationen durchgeführt.
Zu nächst wurde das CNN trainiert, dass es zwischen der Klasse no Tumor und Tumor unterscheidet. Für die Klasse Tumor
wurde die samples der Glioma und Meningioma Klasse zusammengefasst.
Die zweite Klassifikation besteht darin zwischen, dass das CNN unterscheidet, um welche Tumorart es sich handelt.
Dementsprechend wurde die Klasse Glioma und Meningioma betrachtet.\\

Die verwendeten MRT Bilder werden zu beginn auf die Größe $224 \times 224$ Pixel skaliert, aufgrund dessen, dass die Bilder nicht alle die 
gleiche Größe besitzen.
Zudem wurden die Bilder in ein Array umgewandelt und Normiert auf die Werte [0,1].

\section{CNN + Metriken}
Das verwendete CNN besteht aus vier Convolutional Layers. Als Aktivierungsfunktion wurde die ReLU-Funktion verwendet.
Zudem wird ein $3 \times 3$ Kernel und Max-Pooling einer Größe von $2 \times 2$ eingesetzt. 

\section{Hyperparameter}


\section{Trainingssamples reduzieren}


\section{Augmentierung}


\section{Reduzierung einer Sample Klasse}
