\chapter{Diskussion}

In dieser Arbeit wurde die Leistung eines Convolutional Neural Network, unter Verwendung unterschiedlich großer Trainingsdatensätze untersucht.
Dafür wurde Schrittweise die Anzahl an Training samples reduziert. 
Die Leistung wird anhand der Accuracy, Sensitivity und Specificity beurteilt.

Die Ergebnisse zeigten, dass alle Metriken mit zunehmender Sample Anzahl ansteigt. 
Daraus lässt sich schließen, dass die Leistungsfähigkeit des Netzwerks zunimmt, da anhand eines größeren Datensatzes mehr Merkmale und Muster erlernt werden.
Bei der Klassifikation zwischen no Tumor und Tumor ist ab 2042 Training samples eine Sättigung zu beobachten.
Somit lässt sich daraus schließen, dass es mindestens 2042 Samples braucht, damit das Netzwerk eine konstante Leistung zeigt.

Im Vergleich zeigt die Klassifikation zwischen den zwei Tumorarten Glioma und Meningioma keine Eindeutige Sättigung. 
Die Accuracy steigt auf ungefähr \SI{92}{\percent} an.
Die Sensitivity zeigt den Ansatz einer Sättigung, da sie ab 1702 Training samples relativ konstante Werte zeigt.
Und auch die Specificity stabilisiert sich ab 1915 Samples bei ca. \SI{0,92}{}.
Daraus lässt sich schließen, dass das Netzwerk eine zuverlässige Leistung ab 1702 Training samples zeigt.
Jedoch wäre eine größere Datenmenge notwendig, damit die Sättigung der Accuracy deutlich zeigt und sich die Stabilität der Sensitivity und Specificity weiter verbessert. 

Um die Modell Leistung zu verbessern wurde die Augmentation angewendet, welche eingesetz wird um overfitting zu vermeiden.
Für die Klassifikation ist zu sehen, dass die Accuracy und Specificity auch mit Verwendung von Augmentation ihre Sättigung ab 2042 Training samples erreichen.
Die Accuracy beträgt dabei \SI{95.1039}{\percent} und ist somit ca. \SI{1}{\percent} höher, als die Reduzierung der Samples ohne Augmentation.
Die Specificity erreicht die Sättigung schon bei 1702 Training samples.
Die Sensitivity bleibt relativ konstant zwischen 1021 und 1702 Training samples, während bei 1702 Samples, ohne Verwendung von Augmentation, abfällt.
Somit lässt sich sagen, dass die Augmentation einen Leistungsabfall verhindert und im Bereich von niedrigeren Datensatzgrößen die Leistung des Modells konstant bei ungefähr 
\SI{0,89}{} hält.  

Bei der Klassifikation zwischen Glioma und Meningioma 